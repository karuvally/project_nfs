\documentclass[conference]{IEEEtran}
\IEEEoverridecommandlockouts
% The preceding line is only needed to identify funding in the first footnote. If that is unneeded, please comment it out.
\usepackage{cite}
\usepackage{amsmath,amssymb,amsfonts}
\usepackage{algorithmic}
\usepackage{graphicx}
\usepackage{textcomp}
\def\BibTeX{{\rm B\kern-.05em{\sc i\kern-.025em b}\kern-.08em
    T\kern-.1667em\lower.7ex\hbox{E}\kern-.125emX}}
\begin{document}

\title{Enhancing performance and reliability of Network File System\\
{\footnotesize \textsuperscript{*}Note: Sub-titles are not captured in Xplore and
should not be used}
\thanks{Identify applicable funding agency here. If none, delete this.}
}

\author{\IEEEauthorblockN{1\textsuperscript{st} Aswin Babu Karuvally}
\IEEEauthorblockA{\textit{dept. of computer applications} \\
\textit{College of engineering, Trivandrum}\\
Trivandrum, India \\
aswinbabuk@gmail.com}
\and
\IEEEauthorblockN{2\textsuperscript{nd} Basith Hameem}
\IEEEauthorblockA{\textit{dept. of compter applications} \\
\textit{College of Engineering, Trivandrum}\\
Trivandrum, India \\
basithhameem@cet.ac.in}
\and
\IEEEauthorblockN{3\textsuperscript{rd} Ann Jerin Sundar}
\IEEEauthorblockA{\textit{dept. of compter applications} \\
\textit{College of Engineeringm Trivandrum}\\
Trivandrum, India \\
annjerinajs@gmail.com }
\and
\IEEEauthorblockN{4\textsuperscript{th} Prof. John Prakash Joseph}
\IEEEauthorblockA{\textit{dept. of Computer Applications} \\
\textit{College of Engineering, Trivandrum}\\
Trivandrum, India \\
john@cet.ac.in}
} 

\maketitle

\begin{abstract}
Network File System is a commonly used distributed file system, allowing the user to access and manipulate storage on remote computers as if they were part of the local machine. Network File System is notoriously slow in its default configuration. This is accentuated if the NFS environment has more than a dozen clients. When configured to deliver faster speeds by turning on asynchronous mode, the system suffers from higher risk of data corruption and loss.

This paper proposes a number of modifications to the Network File System allowing the file system to provide very high performance, while minimizing the risk of data loss and corruption. Further, the proposed system behaves better in congested networks by consuming less bandwidth ensuring decent speeds even during periods of heavy network traffic.
\end{abstract}

\begin{IEEEkeywords}
UNIX, NFS, Performance, Data loss, Data corruption, Reliability, Optimize, Speed, Tweak, Link, Journal
\end{IEEEkeywords}

\section{Introduction}
Network File System is a distributed File System protocol primarily used by the UNIX family of Operating Systems. It allows users to mount, access and manipulate disk partitions or directories on a remote computer, as if the said partition or directory was a part of the local machine. Network File System was developed as an open standard by SUN Microsystems in 1984.

NFS is widely used in Local Area Networks to conveniently share data and provides users the ability to access their files across the network. Sometimes, a directory access protocol such as LDAP is combined with NFS allowing the users to login to their user account from any computer on the  network.

The main drawback of NFS is the slow read and write speeds it offers with the default setup. Though NFS offers a number of parameters in its configuration files to  increase the performance, these either do not affect the performance much or increases the chance of data corruption and loss. Thus the users are forced to run the system with the default, slow configuration.

In many environments, this leads to lost human productivity as interactive computing becomes impossible and Operating System needing access to data on NFS share often ends up freezing the whole computer. This also bottlenecks the CPU of the computer, thereby wasting precious computing resources.

This paper proposes a number of changes to the Network File System protocol which increases the performance of Network File System while reducing the risk of data corruption and loss. The proposed system also ensures
decent speeds in congested network as it consumes less bandwidth than the original NFS implementation and protects the ability of the NFS server to provide access to files during period of peak network activity.

\section{Background work}

\subsection{Maintaining the Integrity of the Specifications}

The IEEEtran class file is used to format your paper and style the text. All margins, 
column widths, line spaces, and text fonts are prescribed; please do not 
alter them. You may note peculiarities. For example, the head margin
measures proportionately more than is customary. This measurement 
and others are deliberate, using specifications that anticipate your paper 
as one part of the entire proceedings, and not as an independent document. 
Please do not revise any of the current designations.

\section{Research Methodology}
The whole NFS system is tried to benchmark using virtual machine and nfs version 4.2 is used for simulation. 
One of the main reason to use virtual machine is that, they provide a number of networking hardware specifications 
which allow us to simulate target machine network hardware properties. Virtual machine is free to use, fast and easy 
to implement. A 100 Mbps bridged adapter is setup in the virtual machine to directly to outside network. In other 
words NAT (Network Address Translation) is not used because in NAT the translation is done by the CPU. This 
make sure that CPU will not bottleneck the system simulation. Benchmarking the system was a tedious process 
because of the absence of proper benchmarking tool. The benchmarking was properly completed by combing some 
of the available tools. Bonnie ++, Phoronix, Dbench and dd-the command line utility was used. The main 
disadvantages of first three test suits was caching effect. This was effectively avoided by using the command line 
utility "dd". Dbench is a powerful benchmarking tool which itself can simulate upto 512 clients by its own.
Benchmarking was done through proper combination of all these tools.

$ dd if=/dev/zero of=/home/user_1/tmp_file $
$ bsize=1K count=2048000 conv=fdatasync $
 
 The "conv=fdatasync" option flushes each 1 KB of data from cache and this ensures that caching effect on virtual 
 machine and nfs performance is zero.
\section{Working and Typical performance of NFS}
Network File System is a distributed file system originally developed by
SUN Microsystems in 1984. A distributed file system differ from normal file
systems in the sense that they operate over the network and allow sharing of
storage resources. NFS was developed as an open standard. It was initially
implemented for UNIX but is now compatible with a wide array of Operating
Systems. NFS has gone through four major revisions, with the first publicly
available version being v2. All the experiments for this research have been
conducted using NFS v4.2

NFS uses the Client-Server architecture. The NFS Server makes available a
disk partition or directory on the network which can then be mounted just
like a local storage device by the clients. For applications running on the
client machine, NFS is just another file system and requires zero
modifications inorder to work. This is made possible by an abstraction
layer called Virtual File System or VFS. VFS defines what operations can be
done on the filesystem regardless of the file system type. When an
application deals with the UNIX file system, it is actually dealing with
VFS. VFS receives data regarding the target file from the application, then
hands it over to the actual file system, which in this case is NFS.

Once NFS receives the data, it is transfered to the server with the help of
Remote Procedure Call (RPC) in External Data Representation (XDR) format.
RPC allows the NFS client to execute instructions on the server. XDR is a
data representation standard that provides a uniform data format which can
be understood by a variety of computers. This is one of the factors that
provide NFS with cross platform compatibilty. The working of NFS protocol
has been shown in figure.

VFS-NFS handover Figure

NFS by default runs in what is called server side synchronous mode. In this
mode, when the NFS client receives a write operation, it connects to the
server, requsts a write and transfers the data. Once the transfer is
;complete, the server syncs the data to its disks. After completing the sync
operation, server returns an acknowledgment message back to the client. The
issue here is that, the client has to wait till it receives the 
acknowledgment.It cannot perform any additional write to the server till the
acknowledgment arrives. This considerably slows down the system. Further,
clients are often configured to work in client side synchronous mode.
In this mode, the client is forced to write the data to the server as
soon as it receives the request. This often causes the system to crawl.

The performance of an NFS system with client and server side synchronous
mode turned on was benchmarked with client side options in /etc/fstab set as
[rw, sync, hard, intr 0 0] and the server side options  in /etc/exports set
as [rw, no-root-squash, subtree-check]. The benchmarks were conducted using
dbench utility part of Phoronix Test Suite. This resulted in performance of
0.94 MB/s. The server and client systems in this case were equipped with
100mbps network adapters. The speeds obtained are thus clearly sub-optimal.
The peformance graph obtained from the benchmark has been shown in figure.
\section{Parameters affecting performance}
Testing environment
Testing environment is set up in a HP-Z640 Desktop Workstation running RHEL 
(RedHat Enterprise Linux). A NFS server and six clients were setup on a 
virtual machine.Dbench is specifically meant for SMB/NFS benchmarking.
Using the above tools for benchmarking, various tests were run on the 
previously setup environment.There are various factors that can be manipulated 
in the configuration of NFS server side and client side.NFS client side 
configuration is done in "/etc/fstab" whereas NFS server side configuration is 
done in "/etc/exports".
\begin{table}[htbp]
\caption{NFS Options-Client Side}
\begin{center}
\begin{tabular}{|c|c|c|}
\hline
\cline{2-3} 
\textbf{SlNo.} & \textbf{\textit{Option}}& \textbf{\textit{Description}} \\
\hline
1& rw & Read/Write  \\
2& syn & Sync file system with the server  \\
3& hard & NFS requests are retried indefinitely  \\
4& intr & Provided for backward compatibility \\
5& nfsvers & Specifies the nfs versions  \\
6& rsize & Maximum number of bytes when reading data  \\
7& wsize & Maximum number of bytes when writing data  \\
8& udp & Specifies the connection to UDP  \\
9& async & Asynchronous write  \\
\hline
\end{tabular}
\label{tab2}
\end{center}
\end{table}
\begin{table}[htbp]
\caption{NFS Options-Server Side}
\begin{center}
\begin{tabular}{|c|c|c|}
\hline
\cline{2-3} 
\textbf{SlNo.} & \textbf{\textit{Option}}& \textbf{\textit{Description}} \\
\hline
1& rw & Read/Write  \\
2& no-root-squash & Turn off root squashing  \\
3& subtree-check & Specified directory/its subrectory for access  \\
4& async & Synchronous write \\
5& sync & Asynchronous write  \\
\hline
\end{tabular}
\label{tab1}
\end{center}
\end{table}
Various combinations of these options were experimentally simulated in the 
above mentioned environment. Test span was  from 40 minutes to 12 hours. 
Test span depends upon the combination of tools we use and the options we 
enforce in "/etc/fstab" and "/etc/exports". It is found that there is 
considerable decrese in performance when UDP used in client side. Write speed 
decrement 0.94 MB/s to 0.77 MB/s . More importantly there is sharp increase 
in the performance by turning on "async" mode on both server side and client 
side.


\section{Performance with Server side ASYNC}
In server side synchronous mode, the server waits till the data has been 
written to its disk before returning the acknowledgment message to the 
client. Server side asynchronous mode changes the behaviour of NFS server 
such that it returns the acknowledgment message as soon as the client 
completes the transfer of data. This has tremendous impact on the
performance of the system.

An NFS system with server side asynchronous mode was benchmarked with the
client side options in /etc/fstab set as [rw, sync, hard, intr 0 0] and
server side options in /etc/exports set as 
[rw,no-root-squash,no-subtree-check, async]. The test was conducted using
dbench tool - part of phoronix test suite, and resulted in 28.72 MB/s, a
huge increase in performance boost, compared with the earlier server side
synchronous mode, which returned 0.94 MB/s. The performance graph obtained has been shown in figure.
Further benchmarks were also conducted with dd utility, with a block size
of 1KB, file size of 2GB and fdatasync option turned on. The block size of
1K ensures the computer writes only 1KB of data at a time and the fdatasync
option flushes the cache after each block is written. The aim was to
understand the performance of the system without the contribution of client
side caches. The benchmark resulted in 7.1 MB/s which was stil significantly
higher than results recorded with server side asynchronous mode.Performance Figure.
The higher performance comes from the fact that the clients do not have to 
wait till the server syncs the data with its disks. Clients can transfer
data to the server, then get on with other tasks such as writing additional
files. This also means that more number of clients can access the server
in unit time. Still, with the current protocol, it is not advisable to leave
server side async turned on due to the possibility of data loss and
corruption.
\subsection{Reliability concerns with Server side ASYNC}\label{AA}
One side effect of enabling server side asynchronous mode is that, more
clients can access the server in unit time. This in turn can cause a write
queue to form on the client side. That is, writing a file to the server
gives no guarantee that it has been written to permanent storage. In case
the server crashes immediately after some data has been transferred to it,
be it software crash or hardware failure, it can cause data corruption or 
loss. The more worrying fact is that, it is not just a single computer which
looses data. Data loss can occour to most of the clients which has written 
to the server shortly before the crash.

If a server crash occours, a client has no means of protecting itself from
the data loss. A client has no NFS cache that is permanent in nature. Even
if the client has the lost file in its primary memory, there is a high
chance of loosing it. This is because, if prior to crash the server was
serving a critical configuration file, the application dependant on the file
can crash. If the application is part of the Operating System, it can bring
down the whole system. The latter is often the case with environments where
home directory is served by an NFS share.

\subsection{Fixing server side ASYNC behaviour}

Once VFS handovers the write request to NFS client, it transfers the 
received data to the server rather than writing it to the local storage. In
case of data loss, the file cannot be recovered, as the only copy of the
file was in server's memory. The solution is to create a buffer in the 
client's local storage such that, a copy of all the data written to server 
will be kept with the client.

The buffer is a predefined storage area in the client's secondary memory.
It acts like a ring buffer with a flexible memory size. The oldest files are
deleted once the buffer reaches a predefined size. NFS client will maintain
a plaintext file in the buffer containing names of each file in the buffer,
its path and hash generated from each file. During an NFS write operation,
the client stores a copy of the file in the buffer area and updates the 
metadata file with the information regarding the  new file. The hash is
calculated whenever a file is written to the buffer. To minimize the CPU
overhead for hash calculation, one should implement a lightweight hashing
algorithm such as QUARK or PHOTON. Figure shows the working and structure of
the buffer.

Figure of the working and structure of buffer + metadata file

Optional (Sample implementation of buffer deletion strategy)

In case of a server crash, the server creates a list of corrupted or lost
files. During the first boot after the crash, the server requests the
metadata file from each client that connects to it. Once the server receives
the metadata, it calculates the hash for the local copy of each file that
is listed in the metadata. If a file is missing or if the hashes do not
match, they are added to retransmit-list, a list of files to be 
retransmitted from the client. Once the metadata file from a client is fully
scanned, the retransmit list is sent to the client. The client in turn
transmits a new copy of each file in the retransmit list. Figure shows the
error recovery process.

\subsection{Equations}
Number equations consecutively. To make your 
equations more compact, you may use the solidus (~/~), the exp function, or 
appropriate exponents. Italicize Roman symbols for quantities and variables, 
but not Greek symbols. Use a long dash rather than a hyphen for a minus 
sign. Punctuate equations with commas or periods when they are part of a 
sentence, as in:
\begin{equation}
a+b=\gamma\label{eq}
\end{equation}

Be sure that the 
symbols in your equation have been defined before or immediately following 
the equation. Use ``\eqref{eq}'', not ``Eq.~\eqref{eq}'' or ``equation \eqref{eq}'', except at 
the beginning of a sentence: ``Equation \eqref{eq} is . . .''

\subsection{\LaTeX-Specific Advice}

Please use ``soft'' (e.g., \verb|\eqref{Eq}|) cross references instead
of ``hard'' references (e.g., \verb|(1)|). That will make it possible
to combine sections, add equations, or change the order of figures or
citations without having to go through the file line by line.

Please don't use the \verb|{eqnarray}| equation environment. Use
\verb|{align}| or \verb|{IEEEeqnarray}| instead. The \verb|{eqnarray}|
environment leaves unsightly spaces around relation symbols.

Please note that the \verb|{subequations}| environment in {\LaTeX}
will increment the main equation counter even when there are no
equation numbers displayed. If you forget that, you might write an
article in which the equation numbers skip from (17) to (20), causing
the copy editors to wonder if you've discovered a new method of
counting.

{\BibTeX} does not work by magic. It doesn't get the bibliographic
data from thin air but from .bib files. If you use {\BibTeX} to produce a
bibliography you must send the .bib files. 

{\LaTeX} can't read your mind. If you assign the same label to a
subsubsection and a table, you might find that Table I has been cross
referenced as Table IV-B3. 

{\LaTeX} does not have precognitive abilities. If you put a
\verb|\label| command before the command that updates the counter it's
supposed to be using, the label will pick up the last counter to be
cross referenced instead. In particular, a \verb|\label| command
should not go before the caption of a figure or a table.

Do not use \verb|\nonumber| inside the \verb|{array}| environment. It
will not stop equation numbers inside \verb|{array}| (there won't be
any anyway) and it might stop a wanted equation number in the
surrounding equation.

\subsection{Some Common Mistakes}\label{SCM}
\begin{itemize}
\item The word ``data'' is plural, not singular.
\item The subscript for the permeability of vacuum $\mu_{0}$, and other common scientific constants, is zero with subscript formatting, not a lowercase letter ``o''.
\item In American English, commas, semicolons, periods, question and exclamation marks are located within quotation marks only when a complete thought or name is cited, such as a title or full quotation. When quotation marks are used, instead of a bold or italic typeface, to highlight a word or phrase, punctuation should appear outside of the quotation marks. A parenthetical phrase or statement at the end of a sentence is punctuated outside of the closing parenthesis (like this). (A parenthetical sentence is punctuated within the parentheses.)
\item A graph within a graph is an ``inset'', not an ``insert''. The word alternatively is preferred to the word ``alternately'' (unless you really mean something that alternates).
\item Do not use the word ``essentially'' to mean ``approximately'' or ``effectively''.
\item In your paper title, if the words ``that uses'' can accurately replace the word ``using'', capitalize the ``u''; if not, keep using lower-cased.
\item Be aware of the different meanings of the homophones ``affect'' and ``effect'', ``complement'' and ``compliment'', ``discreet'' and ``discrete'', ``principal'' and ``principle''.
\item Do not confuse ``imply'' and ``infer''.
\item The prefix ``non'' is not a word; it should be joined to the word it modifies, usually without a hyphen.
\item There is no period after the ``et'' in the Latin abbreviation ``et al.''.
\item The abbreviation ``i.e.'' means ``that is'', and the abbreviation ``e.g.'' means ``for example''.
\end{itemize}
An excellent style manual for science writers is \cite{b7}.

\subsection{Authors and Affiliations}
\textbf{The class file is designed for, but not limited to, six authors.} A 
minimum of one author is required for all conference articles. Author names 
should be listed starting from left to right and then moving down to the 
next line. This is the author sequence that will be used in future citations 
and by indexing services. Names should not be listed in columns nor group by 
affiliation. Please keep your affiliations as succinct as possible (for 
example, do not differentiate among departments of the same organization).

\subsection{Identify the Headings}
Headings, or heads, are organizational devices that guide the reader through 
your paper. There are two types: component heads and text heads.

Component heads identify the different components of your paper and are not 
topically subordinate to each other. Examples include Acknowledgments and 
References and, for these, the correct style to use is ``Heading 5''. Use 
``figure caption'' for your Figure captions, and ``table head'' for your 
table title. Run-in heads, such as ``Abstract'', will require you to apply a 
style (in this case, italic) in addition to the style provided by the drop 
down menu to differentiate the head from the text.

Text heads organize the topics on a relational, hierarchical basis. For 
example, the paper title is the primary text head because all subsequent 
material relates and elaborates on this one topic. If there are two or more 
sub-topics, the next level head (uppercase Roman numerals) should be used 
and, conversely, if there are not at least two sub-topics, then no subheads 
should be introduced.

\subsection{Figures and Tables}
\paragraph{Positioning Figures and Tables} Place figures and tables at the top and 
bottom of columns. Avoid placing them in the middle of columns. Large 
figures and tables may span across both columns. Figure captions should be 
below the figures; table heads should appear above the tables. Insert 
figures and tables after they are cited in the text. Use the abbreviation 
``Fig.~\ref{fig}'', even at the beginning of a sentence.

\begin{table}[htbp]
\caption{NFS Options-Client Side}
\begin{center}
\begin{tabular}{|c|c|c|}
\hline
\cline{2-3} 
\textbf{SlNo.} & \textbf{\textit{Option}}& \textbf{\textit{Description}} \\
\hline
1& rw & Read/Write  \\
2& syn & Sync file system with the server  \\
3& hard & NFS requests are retried indefinitely  \\
4& intr & Provided for backward compatibility \\
5& nfsvers & Specifies the nfs versions  \\
6& rsize & Maximum number of bytes when reading data  \\
7& wsize & Maximum number of bytes when writing data  \\
8& udp & Specifies the connection to UDP  \\
9& async & Asynchronous write  \\
\hline
\multicolumn{3}{l}{$^{\mathrm{a}}$Sample of a Table footnote.}
\end{tabular}
\label{tab1}
\end{center}
\end{table}

\begin{figure}[htbp]
\centerline{\includegraphics[width=0.8\textwidth,natwidth=610,natheight=642]{fig1.png}}
\caption{Example of a figure caption.}
\label{fig}
\end{figure}

Figure Labels: Use 8 point Times New Roman for Figure labels. Use words 
rather than symbols or abbreviations when writing Figure axis labels to 
avoid confusing the reader. As an example, write the quantity 
``Magnetization'', or ``Magnetization, M'', not just ``M''. If including 
units in the label, present them within parentheses. Do not label axes only 
with units. In the example, write ``Magnetization (A/m)'' or ``Magnetization 
\{A[m(1)]\}'', not just ``A/m''. Do not label axes with a ratio of 
quantities and units. For example, write ``Temperature (K)'', not 
``Temperature/K''.

\section*{Acknowledgment}

The preferred spelling of the word ``acknowledgment'' in America is without 
an ``e'' after the ``g''. Avoid the stilted expression ``one of us (R. B. 
G.) thanks $\ldots$''. Instead, try ``R. B. G. thanks$\ldots$''. Put sponsor 
acknowledgments in the unnumbered footnote on the first page.

\section*{References}

Please number citations consecutively within brackets \cite{b1}. The 
sentence punctuation follows the bracket \cite{b2}. Refer simply to the reference 
number, as in \cite{b3}---do not use ``Ref. \cite{b3}'' or ``reference \cite{b3}'' except at 
the beginning of a sentence: ``Reference \cite{b3} was the first $\ldots$''

Number footnotes separately in superscripts. Place the actual footnote at 
the bottom of the column in which it was cited. Do not put footnotes in the 
abstract or reference list. Use letters for table footnotes.

Unless there are six authors or more give all authors' names; do not use 
``et al.''. Papers that have not been published, even if they have been 
submitted for publication, should be cited as ``unpublished'' \cite{b4}. Papers 
that have been accepted for publication should be cited as ``in press'' \cite{b5}. 
Capitalize only the first word in a paper title, except for proper nouns and 
element symbols.

For papers published in translation journals, please give the English 
citation first, followed by the original foreign-language citation \cite{b6}.

\begin{thebibliography}{00}
\bibitem{b1} G. Eason, B. Noble, and I. N. Sneddon, ``On certain integrals of Lipschitz-Hankel type involving products of Bessel functions,'' Phil. Trans. Roy. Soc. London, vol. A247, pp. 529--551, April 1955.
\bibitem{b2} J. Clerk Maxwell, A Treatise on Electricity and Magnetism, 3rd ed., vol. 2. Oxford: Clarendon, 1892, pp.68--73.
\bibitem{b3} I. S. Jacobs and C. P. Bean, ``Fine particles, thin films and exchange anisotropy,'' in Magnetism, vol. III, G. T. Rado and H. Suhl, Eds. New York: Academic, 1963, pp. 271--350.
\bibitem{b4} K. Elissa, ``Title of paper if known,'' unpublished.
\bibitem{b5} R. Nicole, ``Title of paper with only first word capitalized,'' J. Name Stand. Abbrev., in press.
\bibitem{b6} Y. Yorozu, M. Hirano, K. Oka, and Y. Tagawa, ``Electron spectroscopy studies on magneto-optical media and plastic substrate interface,'' IEEE Transl. J. Magn. Japan, vol. 2, pp. 740--741, August 1987 [Digests 9th Annual Conf. Magnetics Japan, p. 301, 1982].
\bibitem{b7} M. Young, The Technical Writer's Handbook. Mill Valley, CA: University Science, 1989.
\end{thebibliography}

\end{document}
